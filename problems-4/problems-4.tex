\documentclass[12pt, a4paper]{article}
\usepackage{fontspec} % loaded by polyglossia, but included here for transparency 
\defaultfontfeatures{Mapping=tex-text}
\usepackage{polyglossia}
\usepackage{microtype}
\usepackage{hyperref}
\usepackage{xifthen}
\hypersetup{
	colorlinks=true, linkcolor = {black}, urlcolor = {blue}, breaklinks=true
}
\setlength{\parskip}{.5em}
\setmainlanguage{russian} 
\setotherlanguage{english}
\usepackage{enumitem}

% XeLaTeX can use any font installed in your system fonts folder
% Linux Libertine in the next line can be replaced with any 
% OpenType or TrueType font that supports the Cyrillic script.

\newcounter{problem}
\newcommand{\problem}[1]{\refstepcounter{problem}{\bf Задача \theproblem.} \ifthenelse{\isempty{#1}{}}{}{\label{#1}}}

\newfontfamily\russianfont{Times New Roman}
\newfontfamily\englishfont{Times New Roman}
\setmonofont{Courier New}
\newfontfamily{\cyrillicfonttt}{Courier New}

\title{Задания по курсу Python\\Задание 4}
\author{Д.В. Иртегов}
\date{\today}

\begin{document}
\pagestyle{empty}
\maketitle
{\small Задачи необходимо сдать до 26 мая.  Решения необходимо сдавать путем отправки pull request в каталог problems-4 репозитория \\
\url{https://github.com/dmitry-irtegov/NSUPython2018}. 
\\Датой сдачи задания считается дата отправки первого pull request.  Если запрос не принят из-за моих замечаний, у вас есть неделя на их исправление. 

Если запрос принят, задание считается засчитанным.  Если запрос не принят, в комментарии вы можете узнать, почему.

В одном запросе следует отправлять не более одного решения. Если решение состоит из нескольких файлов, в запрос должны быть включены они все.  Все запросы одного студента должны отправляться в каталог с именем, соответствующим его учетной записи.  Например, для задачи 3 из группы задач 2, сдаваемой студентом v-pupkin, рекомендуемое имя файла \verb|problems-3/v-pupkin/task3.py|. }

\problem{} Существует шуточный <<закон Философии>> для статей Википедии: если переходить по первой нормальной ссылке в статье, то рано или поздно мы придем на статью о философии. Ваша задача заключается в том, чтобы проверить этот закон.

Для этого нужно написать программу, которая получает на вход ссылку на статью Википедии, а потом циклически переходит по первой нормальной ссылке и повторяет эту операцию (до тех пор, пока не будет достигнута статья о философии, или ссылки не зациклятся). Нормальной ссылкой будем называть ссылку, которая находится в основном содержании статьи, не в инфобоксах и не в служебных блоках, написана синим цветом (красный соответствует несуществующей статье), не курсивом, не являтся сноской и не находится в скобках. Обратите внимание, что для проверки нормальности не обязательно разбирать таблицы стилей и проверять цвет и т.п., достаточно сделать, чтобы программа работала для текущей верстки википедии (например, можно использовать атрибут class у тегов).

Для удобства проверки сделайте, чтобы последовательность переходов выводилась в стандартный поток вывода.

{\bf Внимание!} Чтобы не создавать большую нагрузку на сервер Википедии, сделайте так, чтобы ваша программа делала не более 2-х запросов в секунду. Для этого воспользуйтесь функцией sleep из модуля time.

\problem{} Вам необходимо написать программу с графическим интерфейсом для моделирования игры «Жизнь» (Conway’s Game of Life). Игра должа загружать начальную конфигурацию из некоторого файла и отображать на экране текущее состояние игры и панель управления моделированием. Должна быть доступна возможность запуска и паузы моделирования, изменения скорости моделирования (как минимум три скорости: 1x, 2x, 4x). Также должна присутствовать возможность редактирования текущего поля в режиме паузы (самостоятельно выберите удобный способ это делать).

Для файлов сохранения рекомендуется использовать форматы \href{http://www.conwaylife.com/wiki/Life_1.06}{Life 1.06} или 
\href{http://www.conwaylife.com/wiki/RLE}{RLE}.

Разрешенные для использования библиотеки: TKinter, PyQt, pygame, Kivy, bitarray.

\problem{} С использованием фреймворка Flask реализуйте набор веб-страниц, обеспечивающих регистрацию пользователей на сайте.  Неаутентифицированный пользователь должен иметь возможность выбрать уникальное имя и задать для него пароль.  Необходимо обеспечить возможность использования русских букв (а лучше, произвольных алфавитных символов Unicode) в именах.  После регистрации, пользователь должен иметь возможность заходить на сайт по паролю и видеть имена тех пользователей, которые зарегистрировались после него, но не тех, кто зарегистрировался раньше.  Неаутентифицированные пользователи не могут видеть никаких имен.  Для хранения данных следует использовать стандартную библиотеку dbm или sqlite.  При первом запуске приложение должно самостоятельно инициализировать файлы БД в текущем каталоге.

Не следует реализовать слишком сложные механизмы аутентификации.  Предполагайте, что в production версии приложения доступ к вашим страницам будет происходить через https, поэтому пароли и куки защищены от перехвата.  Но пароли следует хранить в форме хэшей.

Обратите внимание, что пользователь может ввести в составе имени HTML-теги, код на JavaScript или что-то похожее на них.  Это не следует запрещать, но при показе таких имен необходимо проследить, чтобы они показывались на веб-странице буквально, а не интерпретировались как HTML-код.


\end{document}
